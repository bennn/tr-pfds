\chapter{Introduction}
\label{chap:intro}
%\minitoc

%\section{Functional Data Structures for Typed Racket}

%\section*{Functional Data Structures}

The past decade has seen advancements in the development of efficient
functional data structures, particularly by \cite{oka} and
\cite{bagwell-lists, bagwell-trie}. However, few attempts have been made
to port these structures to
Racket~\footnote{\url{http://planet.plt-scheme.org/display.ss?package=ralist.plt&owner=dvanhorn}
\par ~~~\url{http://planet.plt-scheme.org/display.ss?package=galore.plt&owner=soegaard}}.

In this MS thesis, I present a comprehensive library of efficient
functional data structures in Typed Racket \citep{thf-popl, th-diss}, a
recently developed typed dialect of Racket. The thesis also includes an
investigation of the usefulness of these data structures and evaluation
of the practicality of Typed Racket in the context of functional data
structures.

The programming productivity in a language is largely determined by the
data structures that are built-in to the language. Data structures
available in C-like languages cannot be easily adapted to functional
languages like Typed Racket as they depend crucially on mutation, which
is discouraged and/or disallowed in functional languages. So, functional
language require functional data structures to support functional
programming.

In response, I have developed the library of functional data structures
for Typed Racket

\begin{itemize}
\item
  Variants of FIFO queues including those discussed by \cite{oka} and
  Hood-Melville Queues \citep{hood-mel}.
\item
  Variants of priority queues or heaps which include Binomial Heaps,
  Pairing Heaps \citep{pairing}, Leftist Heap \citep{crane}, Bootstrapped
  Heaps \citep{oka} and Skew Binomial Heap.
\item
  Variants of lists such as random access lists \cite{oka}, catenable
  lists \citep{oka} and vlists \citep{bagwell-lists}
\item
  Other data structures such as Red-Black Trees \cite{oka-red-black},
  Treaps \citep{Seidel}, Hash Lists \citep{bagwell-lists}, Tries
  \citep{bagwell-trie} and Sets.
\end{itemize}

Historically, Racket and Typed Racket provide three data structures:
linked lists, which supports many forms of functional programming,
vectors, and hash tables. However, these are not sufficient. To truly
support efficient programming in a functional style, additional
functional data structure support is required.

%Furtherly, the past decade has seen advancements in the development of
%efficient functional data structures, particularly by \cite{oka} and
%\cite{bagwell-lists, bagwell-trie}. However, few attempts have been made
%to port these structures to Racket \citep{galore, dvh-ra}.% and \citep{}.

%\section*{Typed Racket}
Typed Racket has a novel type checker, based on \emph{occurrence
typing}, that ensures type safety, supports all the Racket idioms and
inter-operates well with Racket. Even though Typed Racket has seen use
in the past several years, all of this use is in classroom settings. The
development of a robust functional data structure library in Typed
Racket, thus provides an opportunity to evaluate the usefulness and
practicality of Typed Racket.
%and extended the evaluation of Typed Racket to more
%real-world projects.

%Second,  Finally, I
%believe that features of Typed Racket are well suited for the developing
%this library.



%\begin{figure}[!htbp]
%  \begin{center}
%    \includegraphics[width=0.9\textwidth]{Chapitre1/arctic_control}
%  \end{center}
%  \caption{A nice image...}
%  \label{fig:jolieImage}
%\end{figure}
%
%\section{An equation}
%
%Just to show argmin and partial derivative commands.
%
%\begin{equation}
%  T = \argmin_T E(T,R,F)
%\end{equation}
%
%Regularization:
%
%\begin{equation}
%  \pd{T}{t} = \Delta T
%\end{equation}
%
%\section{An other section}
%
%Showing a great bullet list environment:
%
%\begin{bulletList}
% \item First point
% \item Second point
%% \item Here is an abbreviation reference \nomenclature{DTI}{Diffusion Tensor Imaging} DTI
%\end{bulletList}
